% Test change for git
%===========================================
% Preamble
\documentclass[11pt]{article}

%**********
%Dependencies
\usepackage[left]{lineno}
\usepackage{titlesec}
\usepackage{amsmath}
\usepackage{amsfonts}
\usepackage{amssymb}
%\usepackage[utf8]{inputenc}
\usepackage{color,soul}
%\usepackage{setspace}
%\usepackage{times}
\usepackage{ogonek}


% Packages from Am. Nat. Template
\usepackage[sc]{mathpazo} %Like Palatino with extensive math support
\usepackage{fullpage}
\usepackage[authoryear,sectionbib,sort]{natbib}
\linespread{1.7}
\usepackage[utf8]{inputenc}
\usepackage{lineno}

% Change default margins
\usepackage{fullpage}


% Change subsection numbering
\renewcommand\thesubsection{\arabic{subsection})}
\renewcommand\thesubsubsection{}

% Subsubsection Title Formatting

\titleformat{\subsubsection}    
{\normalfont\fontsize{12pt}{17}\itshape}{\thesubsubsection}{12pt}{}


% Definitions
\def\mathbi#1{\textbf{\em #1}}


%===========================================

\begin{document}
\title{Two-locus model of sexual antagonism in simultaneous hermaphrodites}
\author{Colin Olito, Justin Cally \& Tim Connallon}
\date{18 July 2016}
\maketitle

update \date{\today}

\section*{Notes}

This project was originally an idea Tim and Crispin Jordan had and which they proposed to collaborate with Jessica K. Abbott on. I have attached their email exchange below for context. Since originally proposing the idea, it was put on the back burner as both Tim and Jordan were quite busy with teaching and other duties (they are both relatively new faculty). In the meantime, Tim and I collaborated on my single-locus model for mutation load and inbreeding depression in simulataneous hermaphrodites, which is currently waiting (to my shame) to be written up as an MS. This set the stage pretty well to follow up with their two-locus idea. Justin Cally is an Honor's student who is working with Tim for a short project (~4wks), and who is providing much needed impetus to the project. \bigskip

\subsection*{Tim's email to Jessica}
	Hi Jessica,
	Sorry for my slowness in getting back in touch ... I talked to Crispin over a week ago, but have been completely distracted by other things since. After chatting, I think he and I are on the same page that we'd both be very interested in following up on a two locus model (i.e., two SA loci) with partial selfing, along the lines of a Patten/Ubeda extension. The motivation is that the effective recombination rate between the loci will decline with the selfing rate (promoting selection on linked allelic combinations), as will the parameter space conducive to maintaining polymorphism at a single locus. The interesting thing is that we don't know which factor should win out, except in the case of strictly additive fitness effects: in that case, we expect that linkage should greatly promote invasion of SA alleles at a locus linked to a segregating SA locus (because the single-locus criteria for balancing selection are not affected by selfing). Under dominance reversal conditions (partially recessive SA costs), it is unclear whether linkage compensates for the otherwise reduced parameter space for balancing selection. So, it would be fun to work this out, and it can probably result in a short manuscript. If this was similar to your thinking, I think it would be a lot of fun for us all to collaborate on it. The only issue for us may be one of timing ... Crispin's time is currently at a premium, but he anticipates an ideal time to move onto this would be in early 2015. (Crispin, please shout out if I got something wrong here). Having played around with some of the dioecious two-locus models, I'm pretty sure analytical results will be possible, assuming we adopt a simple selfing model involving fixed selfing rates per genotype. 

	On the other hand, Jessica, this might not mesh well with your plans or timetable, but please let us know what you think, and we can all make further plans accordingly.

	All the best,
	Tim


\section*{Questions}
\noindent 1) What parameter conditions are necessary for balancing selection on SA alleles at a second locus that is linked to a segregating SA locus? \bigskip

\noindent 2) How does the selfing rate, which reduces effective recombination rate, alter the conditions necessary for balancing selection? This is particularly interesting for conditions other than strictly additive fitness effects ($h_f=h_m=0.5$). \bigskip

\noindent 3) Order of invasion? 
\begin{itemize}
  \item Sequential invasion of antagonistic alleles at two loci (first locus at equilibrium)?
  \item Co-invasion at both loci simultaneously?
\end{itemize}

\newpage{}





\end{document}
