%===========================================
% Preamble

\documentclass[11pt]{article}


%**********
%Dependencies

%**********
%Dependencies
%\usepackage[left]{lineno}
\usepackage{titlesec}
\usepackage{amsmath}
\usepackage{amsfonts}
\usepackage{amssymb}
%\usepackage[utf8]{inputenc}
\usepackage{color,soul}
\usepackage[sc]{mathpazo} %Like Palatino with extensive math support
\usepackage{fullpage}
\usepackage[authoryear,sectionbib,sort]{natbib}
\linespread{1}
\usepackage[utf8]{inputenc}
\usepackage{lineno}
\usepackage[hidelinks]{hyperref}


% New commands: fonts
\newcommand{\code}{\fontfamily{pcr}\selectfont}
\newcommand*\chem[1]{\ensuremath{\mathrm{#1}}}


%===========================================
% Title page


% \title{Cover Letter}
% \author{Colin Olito}
% \date{\today}


\begin{document}
%\maketitle
%\newpage{}


%===========================================
% Document


\section*{}

This study extends previous theory of sexually antagonistic (SA) selection in hermaphrodites by examining the consequences of genetic linkage among SA loci. The primary conclusion is that the reduction in effective recombination rate caused by selfing significantly expands the conditions where SA polymorphism is maintained in partially selfing populations.

%By accounting for linkage in a two-locus context, this study extends previous theory of sexually antagonistic (SA) selection in hermaphrodites to show that SA selection, in addition to recurrent mutation, may provide a plausible mechanism for the maintenance of SA genetic variation in partially selfing populations.

\end{document}