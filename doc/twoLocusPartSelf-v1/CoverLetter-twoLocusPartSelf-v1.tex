%===========================================
% Preamble

\documentclass[11pt]{article}


%**********
%Dependencies

%**********
%Dependencies
%\usepackage[left]{lineno}
\usepackage{titlesec}
\usepackage{amsmath}
\usepackage{amsfonts}
\usepackage{amssymb}
%\usepackage[utf8]{inputenc}
\usepackage{color,soul}
\usepackage[sc]{mathpazo} %Like Palatino with extensive math support
\usepackage{fullpage}
\usepackage[authoryear,sectionbib,sort]{natbib}
\linespread{1}
\usepackage[utf8]{inputenc}
\usepackage{lineno}
\usepackage[hidelinks]{hyperref}


% New commands: fonts
\newcommand{\code}{\fontfamily{pcr}\selectfont}
\newcommand*\chem[1]{\ensuremath{\mathrm{#1}}}


%===========================================
% Title page


% \title{Cover Letter}
% \author{Colin Olito}
% \date{\today}


\begin{document}
%\maketitle
%\newpage{}


%===========================================
% Document


\section*{}
\noindent To the Editor,
\bigskip

Please consider this submission “Consequences of genetic linkage for the maintenance of sexually anatagonistic polymorphism in hermaphrodites” for publication as a Brief Communication article in Evolution.
\bigskip

Although there is a rich body of theory regarding sexually antagonistic (SA) selection for dioecious species, genetic trade-offs between sex functions represent analogous constraints on fitness for hermaphrodites. Indeed, for hermaphrodites, both sex functions must be accommodated by a single phenotype, and there is ample scope for traits with a shared genetic basis to constrain fitness through each sex function. Recent theory based on single-locus models suggests that the maintenance of SA genetic variation should be greatly reduced in partially selfing populations. However, linkage between SA loci has been shown to expand the opportunity for balancing selection in dioecious species. Furthermore, selfing reduces the effective rate of recombination, which should strengthen this effect in selfing populations. However, the consequences of linkage for the maintenance of SA polymorphism in hermaphroditic populations has yet to be explored. Here I develop a two-locus model of SA selection in simultaneous hermaphrodites, and explore the joint influence of linkage, self-fertilization, and dominance on the maintainance of SA polymorphism. 
\bigskip

I find that the reduction in effective recombination rate caused by selfing significantly expands the parameter space where SA polymorphism can be maintained. In particular, linkage facilitates the invasion of male-beneficial alleles, partially compensating for a ``female-bias'' in the net direction of selection created by selfing. This study extends previous theory on SA selection in hermaphrodites by considering the the consequences of SA selection for hermaphrodites in a multi-locus context. Accounting for linkage among SA loci has important implications for not only the maintenance of SA genetic variation, but the evolution of mixed mating systems in hermaphrodites; both of which are long-standing and active questions in evolutionary biology, and will therefore be of interest to a broad readership.
\bigskip

Below, I list some potential Editors and referees, contact details, and a brief summary of their areas of expertise. Please do not hesitate to contact me if you have any questions.
\bigskip

\noindent Thank you for your consideration. \\
\noindent Sincerely,
\bigskip

\noindent Colin Olito \\
\noindent Centre for Geometric Biology, \\
\noindent School of Biological Sciences \\
\noindent Monash University \\
\noindent Melbourne, VIC, 3800 \\
\noindent Australia \\
\noindent colin.olito@gmail.edu \\

\bigskip
\hline
\bigskip

\noindent Suggested Associate Editors: \\
\noindent Dr.~Denis Roze \\
\noindent Dr.~Thomas Bataillon \\
\noindent Dr.~Amy Angert \\
\bigskip

\noindent{} Suggested referees:
\bigskip

\noindent{} Dr.~Manus M. Patten (Georgetown University; E-mail: \url{mmp64@georgetown.edu}). \\
\noindent{} \textit{Expertise}: Theoretical population genetics, with an emphasis on sexual and parental antagonism.
\bigskip

\noindent{} Dr.~Francisco \'{U}beda (University of Tennessee; E-mail: \url{fubeda@utk.edu}). \\
\noindent{} \textit{Expertise}: Population genetics, SA selection, and linkage disequilibrium. 
\bigskip

\noindent{} Dr.~James D. Fry (University of Rochester; E-mail: \url{james.fry@rochester.edu}). \\
\noindent{} \textit{Expertise}: Population genetics, sexual antagonism.
\bigskip

\noindent{} Dr.~Matthew Hartfield (Aarhus University; E-mail: \url{matthew.hartfield@birc.au.dk}). \\
\noindent{} \textit{Expertise}: Theoretical evolutionary biology, with emphasis on hermaphrodites (plants).
\bigskip

\end{document}