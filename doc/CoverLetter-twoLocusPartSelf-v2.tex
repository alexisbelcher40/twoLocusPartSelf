%===========================================
% Preamble

\documentclass[11pt]{article}


%**********
%Dependencies

%**********
%Dependencies
%\usepackage[left]{lineno}
\usepackage{titlesec}
\usepackage{amsmath}
\usepackage{amsfonts}
\usepackage{amssymb}
%\usepackage[utf8]{inputenc}
\usepackage{color,soul}
\usepackage[sc]{mathpazo} %Like Palatino with extensive math support
\usepackage{fullpage}
\usepackage[authoryear,sectionbib,sort]{natbib}
\linespread{1}
\usepackage[utf8]{inputenc}
\usepackage{lineno}
\usepackage[hidelinks]{hyperref}


% New commands: fonts
\newcommand{\code}{\fontfamily{pcr}\selectfont}
\newcommand*\chem[1]{\ensuremath{\mathrm{#1}}}


%===========================================
% Title page


% \title{Cover Letter}
% \author{Colin Olito}
% \date{\today}


\begin{document}
%\maketitle
%\newpage{}


%===========================================
% Document


\section*{}
\noindent To the Editor,
\bigskip

Please consider this revised version of the manuscript “Consequences of genetic linkage for the maintenance of sexually anatagonistic polymorphism in hermaphrodites” (MS 16--0679) which was recently accepted for publication (with minor revision) as a Brief Communication article in Evolution.
\bigskip

I have made several minor changes to the manuscript in response to the comments provided by two reviewers. Most notably, I have added a short paragraph to the discussion regarding the implications of the current model predictions for the genomic location of SA genes in hermaphrodites in response to an interesting point raised by Reviewer 2. I have also modified my language in several places in the Results and Discussion to clarify points that the reviewers raised. Please find my detailed responses to each of the reviewers' comments at the bottom of this cover letter.
\bigskip

\noindent Thank you for your consideration. \\
\noindent Sincerely,
\bigskip

\noindent Colin Olito \\
\noindent Centre for Geometric Biology, \\
\noindent School of Biological Sciences \\
\noindent Monash University \\
\noindent Melbourne, VIC, 3800 \\
\noindent Australia \\
\noindent colin.olito@gmail.edu \\
	
\bigskip
\noindent \rule{\textwidth}{0.4pt}
\bigskip

\newpage{}

\section*{Responses to Editor and Reviewer Comments}


Associate Editor

\noindent Comments to the Author:
\bigskip 

The manuscript explores the scope for the evolution of SA loci in selfing species by considering a 2 locus model and comparing it to predictions made previously using single locus theory and hence ignoring the effect of linkage. The manuscript is concisely and clearly written. Importantly, it also provides enough supplementary material to allow interested readers to look through the derivation of the results obtained.  I think it is a perfect contribution for the "brief communication section". I would encourage the author to produce a final version that takes into consideration the careful comments made by Reviewer 2.

	\begin{quote}
		...
	\end{quote}

%%%%%%%%%%%%%%%%%%%%%%%%%%%%%%%%%%%%%%%
\bigskip
\noindent \rule{8cm}{0.4pt}
\bigskip
%%%%%%%%%%%%%%%%%%%%%%%%%%%%%%%%%%%%%%%

\noindent Reviewer(s)' Comments to Author:
\bigskip

\noindent Reviewer: 1
\bigskip

\noindent Comments
\bigskip

\noindent One major comment: this is an outstanding paper. For context, this is a field I work in and know well, and I am typically not shy with my criticisms of manuscripts that I think can be improved. Kudos to the author.

	\begin{quote}
		Much appreciated. I am glad the reviewer enjoyed reading it.
	\end{quote}

\noindent One minor comment: I think the '14' and '23' in equation (1) want to be subscripted.

	\begin{quote}
		Thank you for catching this mistake in the recursions. I have now subscripted the indexes for $\phi_{ij}$.
	\end{quote}



%%%%%%%%%%%%%%%%%%%%%%%%%%%%%%%%%%%%%%%
\bigskip
\noindent \rule{8cm}{0.4pt}
\bigskip
%%%%%%%%%%%%%%%%%%%%%%%%%%%%%%%%%%%%%%%



\noindent Reviewer: 2
\bigskip

\noindent Comments to the Author
\bigskip

\noindent 1 - Eq.2: is it equivalent to the result obained by Jordan \& Connallon when $C > 0$ ?

	\begin{quote}
		Yes, Eq(2) is equivalent to Eq(8) from Jordan \& Connallon (although they solve for $s_m$ rather than $s_f$). I have changed my wording to make this explicit (\textbf{L.124--127}).
	\end{quote}

\noindent 2 - lines 132 - 134: isn't it possible to find a simple condition for the maintenance of polymorphism when $r=0$, since the system becomes equivalent to a single locus?

	\begin{quote}
		This is a good question, and I have made two changes in order to address it (one to the main text on \textbf{L.132-135}, one to Appendix C). The simple but maybe unsatisfying answer to this question is no. Even assuming complete linkage and additive fitness effects, I couldn't find a way to present the invasion criteria from the third candidate leading eigenvalue in a way that I felt would clarify the results for readers. However, I agree that it is somewhat unsatisfying to just call it complicated and move on, so I have made two changes to the manuscript. First, I have added a brief digression in Appendix C that (1) shows the invasion criteria under complete linkage and additive fitness effects ($r=0$, $h_f=h_m=1/2$); (2) shows the invasion criteria under the added assumption of obligate outcrossing ($C=0$); and (3) highlights that under this second set of assumptions, the invasion criteria are algebraically identical to those given by Eq(11) in Patten \& Haig (2010). Second, I have altered the main text to explain that under the above assumptions, the results of Patten \& Haig (2010) can be recovered, and reference Appendix C so that interested readers may explore these results.
	\end{quote}

\noindent 3 - lines 183 - 185: it doesn't seem obvious to me that SA genes should be more tightly clustered in the genomes of more highly selfing species: I guess one could make the opposite argument that because effective recombination rates are reduced, between-locus interactions can allow the maintenance of polymorphism at more distant loci

	\begin{quote}
		This is a great point, and I think it raises an interesting question about what our expectations should be regarding the genomic location of SA genes in hermaphrodites. I have now replaced this sentence with a short paragraph (\textbf{L.188--198}) to explain why we might expect more or less clustering of SA genes in partial selfers, and to highlight the potential value of an empirical comparison. 
	\end{quote}

\noindent 4 - lines 198 - 201: I don't think that SA selection needs to be invoked to explain the maintenance of traits promoting male gamete performance, as these traits will be selected as long as selfing is not complete.

	\begin{quote}
		...
	\end{quote}

\noindent 5 - line 89: AA $\rightarrow$ AB

	\begin{quote}
		Thanks for catching this. I have now corrected to read $AB$.
	\end{quote}

\noindent 6 - line 95: deterministic

	\begin{quote}
		Thanks again. I have corrected the typo.
	\end{quote}



\end{document}