%%%%%%%%%%%%%%%%%%%%%%%%%%%%%
%Preamble
\documentclass{article}


%Dependencies
\usepackage[left]{lineno}
\usepackage{titlesec}
\usepackage{amsmath}
\usepackage{amsfonts}
\usepackage{amssymb}
\usepackage{color,soul}
\usepackage{ogonek}


% Packages from Am. Nat. Template
\usepackage[sc]{mathpazo} %Like Palatino with extensive math support
\usepackage{fullpage}
\usepackage[authoryear,sectionbib,sort]{natbib}
\linespread{1.7}
\usepackage{lineno}

% Graphics
\usepackage{graphicx}
\graphicspath{{../output/figures/}.pdf}

% New commands: fonts
%\newcommand{\code}{\fontfamily{pcr}\selectfont}
%\newcommand*\chem[1]{\ensuremath{\mathrm{#1}}}

 
%%%%%%%%%%%%%%%%%%%%%%%%%%%%%
% Title Page

\title{Consequences of genetic linkage for the maintenance of sexually anatagonistic polymorphism in simultaneous hermaphrodites}
\author{Colin Olito$^{\ast}$ and Tim Connallon(?)}
\date{\today}

\begin{document}
\maketitle


\noindent{} Centre for Geometric Biology, School of Biological Sciences, Monash University, Victoria 3800, Australia.
\noindent{} $^\ast$ Corresponding author e-mail: colin.olito@gmail.com

\bigskip

\noindent{} \textit{Keywords}: Balancing selection, intralocus sexual conflict, recombination, sexual dimorphism, two-locus model

\bigskip

\noindent{} \textit{Manuscript type}: Brief Communication

\bigskip


% Set line number options
\linenumbers
\modulolinenumbers[1]
\renewcommand\linenumberfont{\normalfont\small}

%%%%%%%%%%%%%%%%%%%%%%%%%%%%%
% Main Text

\newpage{}
\section*{Abstract}

\noindent{} ...

\newpage{}


\section*{Introduction}

\begin{enumerate}
	\item Blah blah blah...
		\begin{enumerate}
			\item Blah blah blah...
			\item Blah blah blah...
				\begin{enumerate}
					\item Blah blah blah...
				\end{enumerate}
		\end{enumerate}

	\item 
		\begin{enumerate}
			\item next point...
			\item next point...
		\end{enumerate}
	
	\item next point...
		\begin{enumerate}
			\item next point...
			\item next point...
		\end{enumerate}
\end{enumerate}



\section*{Model}



\section*{Results}




\section*{Discussion}




\section*{Acknowledgments}
CO thanks C.~Venables for helpful discussions and patience. The ideas in this article were conceived during discussion among Drs.~Tim Connallon, Crispin Jordan, and Jessica K. Abbott. CO gratefully acknowledges their encouragement and feedback in developing these ideas. This work was funded by a Monash University Dean's International Postgraduate Student Scholarship to CO, and by the Monash University Centre for Geometric Biology.

\newpage{}


%%%%%%%%%%%%%%%%%%%%%
% Bibliography
%%%%%%%%%%%%%%%%%%%%%

\begin{thebibliography}{}

\bibitem[{Jordan and Connallon(2014)Jordan and Connallon}]{JordanConnallon2014}
Jordan, C.~Y. and T. Connallon. 2014.
\newblock Sexually antagonistic polymorphism in simultaneous hermaphrodites.
\newblock Evolution 68:3555--3569.

\bibitem[{Patten et~al.(2010)Patten et~al.}]{Patten2010}
M.~M. Patten, D. Haig, and F. \'{U}beda. 2010.
\newblock Fitness variation due to sexual antagonism and linkage disequilibrium.
\newblock Evolution 64:3638--3642.


\end{thebibliography}

\newpage{}




%%%%%%%%%%%%%%%%%%%%%%%%%%%%%%%%%%%%%%%%%%%%%%%%%%%%%%%%%%%%%%%%%%
\section*{Tables}
\renewcommand{\thetable}{\arabic{table}}
\setcounter{table}{0}

\begin{table}[h]
\caption{Two-locus fitness matrix for females ($\mathbf{W}_f$). Rows and columns indicate the haplotype inherited from mothers and fathers respectively.}
\label{Table:Fitness}
\centering
\begin{tabular}{l c c c c} \hline
Haplotype & $y_1 = AB$ & $y_2 = Ab$ & $y_3 = aB$ & $y_4 = ab$ \\
\hline
$x_1 = AB$ & $(1-s_f)^2$ & $(1 - s_f)(1 - h_f s_f)$ & $(1 - s_f)(1 - h_f s_f)$ & $(1 - h_f s_f)^2$ \\
$x_2 = Ab$ & $(1 - s_f)(1 - h_f s_f)$ & $(1-s_f)$ & $(1 - h_f s_f)^2$ & $(1 - h_f s_f)$ \\
$x_3 = aB$ & $(1 - s_f)(1 - h_f s_f)$ & $(1 - h_f s_f)^2$ & $(1-s_f)$ & $(1 - h_f s_f)$ \\
$x_4 = ab$ & $(1 - h_f s_f)^2$ & $(1 - h_f s_f)$ & $(1 - h_f s_f)$ & $1$ \\
\hline
\end{tabular}
\bigskip{}
\end{table}


\newpage{}

\section*{Figures}
 
\begin{figure}[!ht]
\includegraphics[width=\textwidth]{./Fig1}
\caption{Self-fertilization increases the scope for genetic linkage to expand the parameter space where sexually antagonistic polymorphism is maintained. Results are shown for the conditions of additive allelic effects ($h_i = 1/2$; panels A--C), and dominance reversal ($h_i = 1/4$; panels D--F). In each panel, the shaded region between the black lines indicates the region of sexually antagonistic polymorphism for the single locus case; greyscale lines indicate the thresholds for invasion for female-beneficial (lower lines) and male-beneficial (upper lines) alleles at different recombination rates for the two-locus model.}
\label{Fig:Fig1.pdf}
\end{figure}
\newpage{}


\begin{figure}[!ht]
\includegraphics[width=\textwidth]{./Fig2}
\caption{The proportion of parameter space where sexually antagonistic polymorphism is maintained decreases with recombination and self-fertilization (panels A,C). The relative increase in sexually antagonistic polymorphism for the two-locus vs.~single-locus case also decreases with the recombination rate, but increases with the rate of self-fertilization (panels B,D). Results are shown for the conditions of additive allelic effects ($h_i = 1/2$; panels A--B), and dominance reversal ($h_i = 1/4$; panels C--D). Results were obtained by evaluating the leading eigenvalues for 30,000 points distributed uniformly throughout parameter space defined by $s_m \times s_f$.}
\label{Fig:Fig2.pdf}
\end{figure}
\newpage{}




%%%%%%%%%%%%%%%%%%%%%%%%%%%%%%%%%%%%%%%%%%%%%%%%%%%%%%%%%%%%%%%%%%%%%%%
\subsection*{Supporting Information}
\renewcommand{\thefigure}{A\arabic{figure}}
\setcounter{figure}{0}


\begin{figure}[!ht]
\includegraphics[width=\textwidth]{./Fig1wk}
\caption{Under weak selection, the parameter space where sexually antagonistic polymorphism is maintained is only expanded beyond the single-locus case when there is very tight linkage between loci. Thus, there is little scope for self-fertilization to influence sexually antagonistic polymorphism. Results are shown for the conditions of additive allelic effects ($h_i = 1/2$; panels A--C), and dominance reversal ($h_i = 1/4$; panels D--F). In each panel, the shaded region between the black lines indicates the region of sexually antagonistic polymorphism for the single-locus case; grey lines indicate the thresholds for invasion for female-beneficial (lower lines) and male-beneficial (upper lines) alleles under complete linkage ($r = 0$).}
\label{Fig:Fig1wk.pdf}
\end{figure}
\newpage{}

%\begin{figure}[!ht]
%\includegraphics[width=\textwidth]{./figA1}
%\caption{....}
%\label{FigA:figA1.pdf}
%\end{figure}
%\newpage{}



\end{document}
