%%%%%%%%%%%%%%%%%%%%%%%%%%%%%
%Preamble
\documentclass{article}


%Dependencies
\usepackage[left]{lineno}
\usepackage{titlesec}
\usepackage{amsmath}
\usepackage{amsfonts}
\usepackage{amssymb}
\usepackage{color,soul}
\usepackage{ogonek}
\usepackage{float}


% Packages from Am. Nat. Template
\usepackage[sc]{mathpazo} %Like Palatino with extensive math support
\usepackage{fullpage}
\usepackage[authoryear,sectionbib,sort]{natbib}
\linespread{1.7}
\usepackage{lineno}

% Graphics
\usepackage{graphicx}
\graphicspath{{../output/figures/}.pdf}

% New commands: fonts
%\newcommand{\code}{\fontfamily{pcr}\selectfont}
%\newcommand*\chem[1]{\ensuremath{\mathrm{#1}}}
\newcommand\numberthis{\addtocounter{equation}{1}\tag{\theequation}}
 
%%%%%%%%%%%%%%%%%%%%%%%%%%%%%
% Title Page

\title{Consequences of genetic linkage for the maintenance of sexually anatagonistic polymorphism in simultaneous hermaphrodites}
\author{Colin Olito$^{\ast}$ and Tim Connallon(?)}
\date{\today}

\begin{document}
\maketitle


\noindent{} Centre for Geometric Biology, School of Biological Sciences, Monash University, Victoria 3800, Australia.
\noindent{} $^\ast$ Corresponding author e-mail: colin.olito@gmail.com

\bigskip

\noindent{} \textit{Keywords}: Balancing selection, genetic linkage, intralocus sexual conflict, recombination, sexual dimorphism, two-locus model

\bigskip

\noindent{} \textit{Manuscript type}: Brief Communication

\bigskip


% Set line number options
\linenumbers
\modulolinenumbers[1]
\renewcommand\linenumberfont{\normalfont\small}

%%%%%%%%%%%%%%%%%%%%%%%%%%%%%
% Main Text

\newpage{}
\section*{Abstract}

\noindent{} ...

\newpage{}


\section*{Introduction}

\begin{enumerate}
	\item What are SA alleles...
		\begin{enumerate}
			\item Scope for SA selection to maintain polymorphism in a variety of contexts.
			\item Trade-offs between the sex-specific fitness components of hermaphrodites are roughly analogous to SA trade-offs in dioecious species (Abbott 2011), and for simplicity, we retain the use of the term “SA” when referring to genetic variation that causes a fitness trade-off between the male and female functions of hermaphrodites
			\item Previous emphasis on dioecious species in SA literature, yet fitness trade-offs between sex functions can also affect the evolution of hermaphrodite populations in which “male” and “female” sex functions jointly contribute to each individual’s total fitness (Lloyd and Webb 1986; Webb and Lloyd 1986; Barrett 2002; Abbott 2011)
				\begin{enumerate}
					\item Blah blah blah...
					\item Blah blah blah...
				\end{enumerate}
		\end{enumerate}

	\item In dioecious models, tight linkage between SA loci ($r < 0.2$) can maintain variants that are lost in single-locus models of SA selection, expanding the parameter space where SA polymorphism is maintained.

	\item For partially-selfing hermaphrodites, single-locus models of SA predict that...
		\begin{enumerate}
			\item Increased selfing diminishes the opportunity for SA polymorphism to be maintained and...
			\item decreases the sensitivity of SA polymorphism to dominanace and..
			\item Skews the parameter space where SA polymorphism is predicted, increasing the opportunity for female-beneficial alleles to invade, while decreasing the opportunity for male-beneficial alleles to invade.
		\end{enumerate}
	
	\item However, increased selfing also decreases the effective recombination rate among nearby loci, promoting selection on linked allelic combinations.
		\begin{enumerate}
			\item It is unclear which factor should be more important in determining whether SA polymorphism is maintained, or how their relative importance may change under different dominance conditions (i.e. additive vs. ``dominance-reversal'')
			\item Under dominance reversal conditions (partially recessive SA costs), it is unclear whether linkage compensates for the otherwise reduced parameter space where balancing selection is predicted.
		\end{enumerate}
\end{enumerate}

Here I investigate the joint influence of genetic linkage, partial selfing, and dominance on opportunities for maintaining SA polymorphism in simultaneous hermaphrodites. 


\section*{Model}

Consider a genetic system involving two diallelic autosomal loci $\mathbf{A}$ and $\mathbf{B}$ that recombine at a rate $r$ in an infinite population of simultaneous hermaphrodites. The rate of self-fertilization ($C$) in the population is independent of the genotype at the loci in question (a `fixed' selfing model \textit{sensu} \citealt{Holden1979,CaballeroHill1992, JordanConnallon2014}). Generations are assumed to be discrete and non-overlapping, with selection occuring on diploid adults before mating. Let $x_i$ and $y_i$ denote the frequencies of the four possible haplotypes $[AB, Ab, aB, ab]$ among male and female gametes respectively. Both loci are under sexually antagonistic selection such that $A$ and $B$ represent male-beneficial alleles, while $a$ and $b$ represent female-beneficial alleles. That is, the fitness of females with with genotypes $AA$, $Aa$, and $aa$ at the $\mathbf{A}$ locus are equal to $1 - s_f$, $1 - h_f s_f$, and $1$ (after \citealt{Kidwell1977}). Fitness at the $\mathbf{B}$ locus follows the same parameterization. The fitness of offspring formed by the union of the $i$th female and $j$th male gametic, $w_{k,ij}$ (where $k \in [m,f]$), haplotypes is assumed to be equal to the product of the fitnesses at $\mathbf{A}$ and $\mathbf{B}$ (Table~\ref{Table:Fitness}). Following convention for SA models (e.g.~\citealt{Kidwell1977}), sex-specific selection coefficients are constrained to be $0 < s_k < 1$.

Genotype frequencies in the next generation are given by a system of ten recursion equations (\citealt{Holden1979, JordanConnallon2014}). However, it is possible to approximate the evolutionary trajectories of haplotypes in a sexually antagonistic two-locus system under weak selection (See Appendix A in Supplementary Information). If selection is assumed to be relatively weak, the rate of allele frequency change due to selection should be slow relative to the rate at which genotypes approach equilibirium in partially selfing populations (\citealt{Nagylaki1997}). Under these conditions, it may be appropriate to use a separation of timescales (\citealt{OttoDay2007}), and calculate the quasi-equilibrium (QE) genotypic frequencies in the absence of selection. The genotypic recursions of allele frequency change across generations can then be approximated by substituting into them the QE frequencies, yielding a reduced system of only four haplotype recursions. Using this approach, I model the evolution of the four-haplotype system $q_i = [AA, Ab, aB, ab]$, where the QE adult genotypic frequencies are denoted by $\phi_{ij}$. The recursions giving the haplotype frequencies in the next generation are then

\begin{align*} \label{eq:QEhapRec}
	q'_1 &\approx (1 - C) \frac{(x_1 + y_1)}{2} + C \bigg( \frac{x_1 - r(\phi{14} - \phi{23})}{2 \overline{w}_f} \bigg) \\
	q'_2 &\approx (1 - C) \frac{(x_2 + y_2)}{2} + C \bigg( \frac{x_2 + r(\phi{14} - \phi{23})}{2 \overline{w}_f} \bigg) \\
	q'_3 &\approx (1 - C) \frac{(x_3 + y_3)}{2} + C \bigg( \frac{x_3 + r(\phi{14} - \phi{23})}{2 \overline{w}_f} \bigg) \\
	q'_4 &\approx (1 - C) \frac{(x_4 + y_4)}{2} + C \bigg( \frac{x_4 - r(\phi{14} - \phi{23})}{2 \overline{w}_f} \bigg). \numberthis
\end{align*}

\noindent{} where $x_i$ and $y_i$ are functions $f(C, s_f, s_m, \phi_{ij})$, and $\overline{w}_f$ is the population average fitness through female function (Appendix A in Supplementary Information).

To identify the parameter conditions under which SA polymorphism is predicted to be maintained, I evaluated the leading eigenvalue, $\lambda$, of the Jacobian, $\mathbf{J}$, of the system Eq(\ref{eq:QEhapRec}) for populations initially fixed for the $[AB]$ or $[ab]$ haplotypes respectively (\citealt{Prout1968, OttoDay2007}). Under these conditions, one of two eigenvalues was always the leading eigenvalue of $\mathbf{J}$. The first, $\lambda_1$, describes the rate of change of the system when invasion occurs at a single locus, and yields identical invasion criteria to an equivalent single-locus SA model with four alleles. $\lambda_2$ involves the recombination rate $r$, and identifies the regions of parameter space where invasion of the rare allele is influenced by linkage. General invasion criteria were found by solving for the conditions where $\lambda > 1$ using relevant values for other population genetic parameters. In particular, I focus on representative dominance conditions corresponding to additive allelic effects ($h_f = h_m = 1/2$), and partially recessive allelic effects yielding a ``dominance reversal'' ($h_f = h_m = 1/4$) (\citealt{Kidwell1977, Prout2000, Fry2010, JordanConnallon2014}).





\section*{Results}




\section*{Discussion}




\section*{Acknowledgments}
CO thanks J.~Cally, C.~Venables, and T.~Connallon for helpful discussions. The idea for this article was conceived during discussion among Drs.~Tim Connallon, Crispin Jordan, and Jessica K. Abbott, and CO gratefully acknowledges their encouragement and feedback in developing it fully. This work was funded by a Monash University Dean's International Postgraduate Student Scholarship to CO, and by the Monash University Centre for Geometric Biology.

\newpage{}





%%%%%%%%%%%%%%%%%%%%%%%%%%%%%%%%%%%%%%%%%%%%%%%%%%%%%%%%%%%%%%%%%%
\section*{Tables}
\renewcommand{\thetable}{\arabic{table}}
\setcounter{table}{0}

\begin{table}[h]
\caption{Two-locus fitness matrix for females ($\mathbf{W}_f$). Rows and columns indicate the haplotype inherited from mothers and fathers respectively. The corresponding fitness matrix for males ($\mathbf{W}_m$) is easily constructed.}
\label{Table:Fitness}
\centering
\begin{tabular}{l c c c c} \hline
Haplotype & $y_1 = AB$ & $y_2 = Ab$ & $y_3 = aB$ & $y_4 = ab$ \\
\hline
$x_1 = AB$ & $(1-s_f)^2$ & $(1 - s_f)(1 - h_f s_f)$ & $(1 - s_f)(1 - h_f s_f)$ & $(1 - h_f s_f)^2$ \\
$x_2 = Ab$ & $(1 - s_f)(1 - h_f s_f)$ & $(1-s_f)$ & $(1 - h_f s_f)^2$ & $(1 - h_f s_f)$ \\
$x_3 = aB$ & $(1 - s_f)(1 - h_f s_f)$ & $(1 - h_f s_f)^2$ & $(1-s_f)$ & $(1 - h_f s_f)$ \\
$x_4 = ab$ & $(1 - h_f s_f)^2$ & $(1 - h_f s_f)$ & $(1 - h_f s_f)$ & $1$ \\
\hline
\end{tabular}
\bigskip{}
\end{table}


\newpage{}

\section*{Figures}
 
\begin{figure}[H]
\includegraphics[scale=0.8]{./Fig1}
\caption{Self-fertilization increases the scope for genetic linkage to expand the parameter space where sexually antagonistic polymorphism is maintained. Results are shown for the conditions of additive allelic effects ($h_i = 1/2$; panels A--C), and dominance reversal ($h_i = 1/4$; panels D--F). In each panel, the shaded region between the black lines indicates the region of sexually antagonistic polymorphism for the single locus case; greyscale lines indicate the thresholds for invasion for female-beneficial (lower lines) and male-beneficial (upper lines) alleles at different recombination rates for the two-locus model.}
\label{Fig:Fig1.pdf}
\end{figure}
\newpage{}


\begin{figure}[H]
\includegraphics[scale=0.9]{./Fig2}
\caption{The proportion of parameter space where sexually antagonistic polymorphism is maintained declines with increased recombination and self-fertilization (panels A,C). The increase in sexually antagonistic polymorphism in the two-locus model relative to the single-locus case also declines with the recombination rate, but increases with the rate of self-fertilization (panels B,D). Results are shown for the conditions of additive allelic effects ($h_i = 1/2$; panels A--B), and dominance reversal ($h_i = 1/4$; panels C--D). Results were obtained by evaluating the leading eigenvalues (calculated from Eqn.~\hl{X}) for 30,000 points distributed uniformly throughout parameter space defined by $s_m \times s_f$.}
\label{Fig:Fig2.pdf}
\end{figure}
\newpage{}




%%%%%%%%%%%%%%%%%%%%%%%%%%%%%%%%%%%%%%%%%%%%%%%%%%%%%%%%%%%%%%%%%%%%%%%
\section*{Supporting Information}

\subsection*{Appendix A: Development of recursions, and quasi-equilibrium approximations of allele frequency change}
\renewcommand{\theequation}{A\arabic{equation}}
\titleformat{\subsubsection}    
{\normalfont\fontsize{12pt}{17}\itshape}{\thesubsubsection}{12pt}{}

\subsubsection*{Recursions}

Consider the genetic system described in the main text. Adult genotypic frequencies formed by the union of the $i$th and $j$th haplotypes are denoted using $F_{ij}$, and the overall haplotype frequencies, $q_i$ are

\begin{align*}
	q_1 &= F_{11} + \frac{F_{12} + F_{13} + F_{14}}{2} \\
	q_2 &= F_{22} + \frac{F_{12} + F_{23} + F_{24}}{2} \\
	q_3 &= F_{33} + \frac{F_{13} + F_{23} + F_{34}}{2} \\
	q_4 &= F_{44} + \frac{F_{14} + F_{24} + F_{34}}{2}. \numberthis \label{eqn}
\end{align*} 

Haplotype frequencies in female gametes are denoted $x_1 = [AB]$, $x_2 = [Ab]$, $x_3 = [aB]$, and $x_4 = [ab]$, and the corresponding frequencies in male gametes are denoted $y_1$, $y_2$, $y_3$, and $y_4$. The contribution of genotype $ij$ to the production of offspring in the next generation is denoted by $F^f_{ij}$. Given these conditions, the recursion equations for the genotypic frequencies in the next generation are (\citealt{Holden1979, JordanConnallon2014})

\begin{align*} \label{eq:genRec}
	F'_{11} &= (1 - C) (x_1 y_1)           + C \bigg( F^f_{11} + \frac{F^f_{12} + F^f_{13} + F^f_{14}(1 - r)^2 + F^f_{23} r^2}{4} \bigg)  \\
	F'_{12} &= (1 - C) (x_1 y_2 + x_2 y_1) + C \bigg( \frac{F^f_{12} + F^f_{14} r(1-r) + F^f_{23} r(1-r)}{2} \bigg)  \\
	F'_{13} &= (1 - C) (x_1 y_3 + x_3 y_1) + C \bigg( \frac{F^f_{13} + F^f_{14} r(1-r) + F^f_{23} r(1-r)}{2} \bigg)  \\
	F'_{14} &= (1 - C) (x_1 y_4 + x_4 y_1) + C \bigg( \frac{F^f_{14} (1-r)^2 + F^f_{23} r^2}{2} \bigg)  \\
	F'_{22} &= (1 - C) (x_2 y_2)           + C \bigg( F^f_{22} + \frac{F^f_{12} + F^f_{14} r^2 + F^f_{23} (1-r)^2 + F^f_{24}}{4} \bigg)  \\
	F'_{23} &= (1 - C) (x_2 y_3 + x_3 y_2) + C \bigg( \frac{F^f_{14} r^2 + F^f_{23} (1-r)^2}{2} \bigg)  \\
	F'_{24} &= (1 - C) (x_2 y_4 + x_4 y_2) + C \bigg( \frac{F^f_{24} + F^f_{14} r(1-r) + F^f_{23} r(1-r)}{2} \bigg)  \\
	F'_{33} &= (1 - C) (x_3 y_3)           + C \bigg( F^f_{33} + \frac{F^f_{13} + F^f_{14} r^2 + F^f_{23} (1-r)^2 + F^f_{34}}{4} \bigg)  \\
	F'_{34} &= (1 - C) (x_3 y_4 + x_4 y_3) + C \bigg( \frac{F^f_{14} r(1-r) + F^f_{23} r(1-r) + F^f_{34}}{2} \bigg)  \\
	F'_{44} &= (1 - C) (x_4 y_4)           + C \bigg( F^f_{44} + \frac{F^f_{14} (1-r)^2 + F^f_{23} r^2 + F^f_{24} + F^f_{34}}{4} \bigg)  \numberthis
\end{align*}

\noindent{} where $x_{i}$ are

\begin{align*}
	x_1 &= \frac{2 F_{11} w_{f11} + F_{12} w_{f12} + F_{13} w_{f13} + F_{14} w_{f14}}{2 \overline{w}_f} - r \bigg( \frac{F_{14} w_{f14} - F_{23} w_{f14}}{2 \overline{w}_f} \bigg) \\
	x_2 &= \frac{2 F_{22} w_{f22} + F_{12} w_{f12} + F_{23} w_{f23} + F_{24} w_{f24}}{2 \overline{w}_f} + r \bigg( \frac{F_{14} w_{f14} - F_{23} w_{f23}}{2 \overline{w}_f} \bigg) \\
	x_3 &= \frac{2 F_{33} w_{f33} + F_{34} w_{f34} + F_{13} w_{f13} + F_{23} w_{f23}}{2 \overline{w}_f} + r \bigg( \frac{F_{14} w_{f14} - F_{23} w_{f23}}{2 \overline{w}_f} \bigg) \\
	x_4 &= \frac{2 F_{44} w_{f44} + F_{32} w_{f32} + F_{14} w_{f14} + F_{24} w_{f24}}{2 \overline{w}_f} - r \bigg( \frac{F_{14} w_{f14} - F_{23} w_{f23}}{2 \overline{w}_f} \bigg), \numberthis
\end{align*}

\noindent{} $y_{i}$ are

\begin{align*}
	y_1 &= \frac{2 F_{11} w_{m11} + F_{12} w_{m12} + F_{13} w_{m13} + F_{14} w_{m14}}{2 \overline{w}_m} - r \bigg( \frac{F_{14} w_{m14} - F_{23} w_{m14}}{2 \overline{w}_m} \bigg) \\
	y_2 &= \frac{2 F_{22} w_{m22} + F_{12} w_{m12} + F_{23} w_{m23} + F_{24} w_{m24}}{2 \overline{w}_m} + r \bigg( \frac{F_{14} w_{m14} - F_{23} w_{m23}}{2 \overline{w}_m} \bigg) \\
	y_3 &= \frac{2 F_{33} w_{m33} + F_{34} w_{m34} + F_{13} w_{m13} + F_{23} w_{m23}}{2 \overline{w}_m} + r \bigg( \frac{F_{14} w_{m14} - F_{23} w_{m23}}{2 \overline{w}_m} \bigg) \\
	y_4 &= \frac{2 F_{44} w_{m44} + F_{32} w_{m32} + F_{14} w_{m14} + F_{24} w_{m24}}{2 \overline{w}_m} - r \bigg( \frac{F_{14} w_{m14} - F_{23} w_{m23}}{2 \overline{w}_m} \bigg), \numberthis
\end{align*}

\noindent{} where ${w}_{k,ij}$ represent the fitness through each sex function ($k \in [m,f]$) of adults resulting from the union of the $i$th and $j$th haplotypes, and the population mean fitness through each sex function is

\begin{align*}
	\overline{w}_f = F_{11} w_{f11} + &F_{12} w_{f12} + F_{13} w_{f13} + F_{14} w_{f14} + F_{22} w_{f22} + \\ 
				     &F_{23} w_{f23} + F_{24} w_{f24} + F_{33} w_{f33} + F_{34} w_{f34} + F_{44} w_{f44} \numberthis \\
	\overline{w}_m = F_{11} w_{m11} + &F_{12} w_{m12} + F_{13} w_{m13} + F_{14} w_{m14} + F_{22} w_{m22} + \\
				     &F_{23} w_{m23} + F_{24} w_{m24} + F_{33} w_{m33} + F_{34} w_{m34} + F_{44} w_{m44}. \numberthis
\end{align*}


Combining Eqns(A1-A6) and simplifying yields the general haplotype recursion equations: 

\begin{align*} \label{eq:hapRec}
	q'_1 &= F'_{11} + \frac{F'_{12} + F'_{13} + F'_{14}}{2} = (1 - C) \frac{(x_1 + y_1)}{2} + C \bigg( \frac{x_1 - r(F_{14} - F_{23})}{2 \overline{w}_f} \bigg) \\
	q'_2 &= F'_{22} + \frac{F'_{12} + F'_{23} + F'_{24}}{2} = (1 - C) \frac{(x_2 + y_2)}{2} + C \bigg( \frac{x_2 + r(F_{14} - F_{23})}{2 \overline{w}_f} \bigg) \\
	q'_3 &= F'_{33} + \frac{F'_{13} + F'_{23} + F'_{34}}{2} = (1 - C) \frac{(x_3 + y_3)}{2} + C \bigg( \frac{x_3 + r(F_{14} - F_{23})}{2 \overline{w}_f} \bigg) \\
	q'_4 &= F'_{44} + \frac{F'_{14} + F'_{24} + F'_{34}}{2} = (1 - C) \frac{(x_4 + y_4)}{2} + C \bigg( \frac{x_4 - r(F_{14} - F_{23})}{2 \overline{w}_f} \bigg). \numberthis
\end{align*}


\subsubsection*{QE Approximations}

After \citet{CaballeroHill1992}, the QE genotypic frequencies for a single locus, $\mathbf{A}$, are

\begin{align*}
	F_{AA}^* &= q^2 + \frac{C q(1 - q)}{(2 - C)} \\
	F_{Aa}^* &= 2 q (1 - q) - \frac{2 C q(1 - q)}{(2 - C)} \\
	F_{aa}^* &= (1 - q)^2 + \frac{C q(1 - q)}{2 - C}, \numberthis 
\end{align*}

\noindent{} where $q$ is the allele frequency and $C$ is the fixed proportion of self fertilization. We can extend these single-locus results for the two-locus model to calculate the QE adult genotypic frequencies formed by the union of the $i$th and $j$th haplotypes:

\begin{align*}
	\phi_{11} &= q_1^2 + C q_1 (1 - q_1) / (2-C)  \\
	\phi_{12} &= 4 q_1 q_2 (1 - C) / (2-C)        \\
	\phi_{13} &= 4 q_1 q_3 (1 - C) / (2-C)        \\
	\phi_{14} &= 4 q_1 q_4 (1 - C) / (2-C)        \\
	\phi_{22} &= q_2^2 + C q_2 (1 - q_2) / (2-C)  \\
	\phi_{23} &= 4 q_2 q_3 (1 - C) / (2-C)        \\
	\phi_{24} &= 4 q_2 q_4 (1 - C) / (2-C)        \\
	\phi_{33} &= q_3^2 + C q_3 (1 - q_3) / (2-C)  \\
	\phi_{34} &= 4 q_3 q_4 (1 - C) / (2-C)        \\
	\phi_{44} &= q_4^2 + C q_4 (1 - q_4) / (2-C)  \numberthis
\end{align*}

Following standard two-locus theory for partially selfing populations (\citealt{Holden1979, OttoDay2007, JordanConnallon2014}), and substituting the QE genotypic frequencies, $\phi_{ij}$, yields the haplotype recursions given in Eqn(\ref{eq:QEhapRec}), where $x_{i}$ are

\begin{align*}
	x_1 &= \frac{2 \phi_{11} w_{f11} + \phi_{12} w_{f12} + \phi_{13} w_{f13} + \phi_{14} w_{f14}}{2 \overline{w}_f} -
				r \bigg( \frac{\phi_{14} w_{f14} - \phi_{23} w_{f23}} {2 \overline{w}_f} \bigg) \\
	x_2 &= \frac{2 \phi_{22} w_{f22} + \phi_{12} w_{f12} + \phi_{23} w_{f23} + \phi_{24} w_{f24}}{2 \overline{w}_f} +
				r \bigg( \frac{\phi_{14} w_{f14} - \phi_{23} w_{f23}} {2 \overline{w}_f} \bigg) \\ 
	x_3 &= \frac{2 \phi_{33} w_{f33} + \phi_{34} w_{f34} + \phi_{13} w_{f13} + \phi_{23} w_{f23}}{2 \overline{w}_f} +
				r \bigg( \frac{\phi_{14} w_{f14} - \phi_{23} w_{f23}} {2 \overline{w}_f} \bigg) \\
	x_4 &= \frac{2 \phi_{44} w_{f44} + \phi_{34} w_{f34} + \phi_{13} w_{f13} + \phi_{24} w_{f24}}{2 \overline{w}_f} -
				r \bigg( \frac{\phi_{14} w_{f14} - \phi_{23} w_{f23}} {2 \overline{w}_f} \bigg) \numberthis
\end{align*}

\noindent{} and $y_{i}$ are

\begin{align*}
	y_1 &= \frac{2 \phi_{11} w_{m11} + \phi_{12} w_{m12} + \phi_{13} w_{m13} + \phi_{14} w_{m14}}{2 \overline{w}_m} -
				r \bigg( \frac{\phi_{14} w_{m14} - \phi_{23} w_{m23}} {2 \overline{w}_m} \bigg) \\
	y_2 &= \frac{2 \phi_{22} w_{m22} + \phi_{12} w_{m12} + \phi_{23} w_{m23} + \phi_{24} w_{m24}}{2 \overline{w}_m} +
				r \bigg( \frac{\phi_{14} w_{m14} - \phi_{23} w_{m23}} {2 \overline{w}_m} \bigg) \\
	y_3 &= \frac{2 \phi_{33} w_{m33} + \phi_{34} w_{m34} + \phi_{13} w_{m13} + \phi_{23} w_{m23}}{2 \overline{w}_m} +
				r \bigg( \frac{\phi_{14} w_{m14} - \phi_{23} w_{m23}} {2 \overline{w}_m} \bigg) \\
	y_4 &= \frac{2 \phi_{44} w_{m44} + \phi_{34} w_{m34} + \phi_{13} w_{m13} + \phi_{24} w_{m24}}{2 \overline{w}_m} -
				r \bigg( \frac{\phi_{14} w_{m14} - \phi_{23} w_{m23}} {2 \overline{w}_m} \bigg) \numberthis
\end{align*}

\noindent{} and the population mean fitness through each sex function is

\begin{align*}
	\overline{w}_f = \phi_{11} w_{f11} + &\phi_{12} w_{f12} + \phi_{13} w_{f13} + \phi_{14} w_{f14} + \phi_{22} w_{f22} + \\ 
				     &\phi_{23} w_{f23} + \phi_{24} w_{f24} + \phi_{33} w_{f33} + \phi_{34} w_{f34} + \phi_{44} w_{f44} \numberthis \\
	\overline{w}_m = \phi_{11} w_{m11} + &\phi_{12} w_{m12} + \phi_{13} w_{m13} + \phi_{14} w_{m14} + \phi_{22} w_{m22} + \\
				     &\phi_{23} w_{m23} + \phi_{24} w_{m24} + \phi_{33} w_{m33} + \phi_{34} w_{m34} + \phi_{44} w_{m44}. \numberthis
\end{align*}










\subsection*{Appendix B: }
\renewcommand{\thefigure}{B\arabic{figure}}
\setcounter{figure}{0}

test





\subsection*{Appendix C: Supplementary figures}
\renewcommand{\thefigure}{C\arabic{figure}}
\setcounter{figure}{0}

\begin{figure}[H]
\includegraphics[scale=0.75]{./Fig1wk}
\caption{Under weak selection, the parameter space where sexually antagonistic polymorphism is maintained is only expanded beyond the single-locus case when there is very tight linkage ($r \approx 0$). Thus, there is little scope for self-fertilization to influence sexually antagonistic polymorphism. Results are shown for the conditions of additive allelic effects ($h_i = 1/2$; panels A--C), and dominance reversal ($h_i = 1/4$; panels D--F). In each panel, the shaded region between the black lines indicates the region of sexually antagonistic polymorphism for the single-locus case; grey lines indicate the thresholds for invasion for female-beneficial (lower lines) and male-beneficial (upper lines) alleles under complete linkage ($r = 0$).}
\label{Fig:Fig1wk.pdf}
\end{figure}
\newpage{}

\begin{figure}[H]
\includegraphics[scale=0.7]{./recSimFig_add}
\caption{Haplotype recursions evaluated at QE, and the resulting invasion conditions, approximate the evolutionary trajectory of the full genotypic recursions very well under additive allelic effects, even under strong selection. Predicted regions of SA polymorphism based on deterministic simulations using the genotypic recursions Eqns(\ref{eq:genRec}) are compared against the outcome of the invasion analysis for the haplotype recursions using the QE approximation (Eqns~\ref{eq:QEhapRec}) across a gradient of selfing ($C$) and recombination ($r$) rates. Green points indicate parameter conditions where the genotypic recursions and the QE haplotype approximations both predicted polymorphism. Red points indicate regions where the genotypic recursions predicted polymorphism but the QE haplotype recursions did not; and blue points indicate the opposite. The proportion of each outcome is shown in the upper left corner of each panel. Black solid lines show the invasion conditions based on the haplotype recursions using the QE approximation for the given values of $C$ and $r$, with lines drawn for both the two-locus and single-locus invasion conditions when $r > 0$.}
\label{Fig:recSimFig_add.pdf}
\end{figure}
\newpage{}


\begin{figure}[H]
\includegraphics[scale=0.7]{./recSimFig_domRev}
\caption{Haplotype recursions evaluated at QE, and the resulting invasion conditions, approximate the evolutionary trajectory of the full genotypic recursions very well under allelic effects resulting in dominance reversal, even under strong selection. Predicted regions of SA polymorphism based on deterministic simulations using the genotypic recursions Eqns(\ref{eq:genRec}) are compared against the outcome of the invasion analysis for the haplotype recursions using the QE approximation (Eqns~\ref{eq:QEhapRec}) across a gradient of selfing ($C$) and recombination ($r$) rates. Green points indicate parameter conditions where the genotypic recursions and the QE haplotype approximations both predicted polymorphism. Red points indicate regions where the genotypic recursions predicted polymorphism but the QE haplotype recursions did not; and blue points indicate the opposite. The proportion of each outcome is shown in the upper left corner of each panel. Black solid lines show the invasion conditions based on the haplotype recursions using the QE approximation for the given values of $C$ and $r$, with lines drawn for both the two-locus and single-locus invasion conditions when $r > 0$.}
\label{Fig:recSimFig_domRev.pdf}
\end{figure}
\newpage{}



%%%%%%%%%%%%%%%%%%%%%
% Bibliography
%%%%%%%%%%%%%%%%%%%%%

\begin{thebibliography}{}

\bibitem[{Caballero and Hill(1992)Caballero and Hill}]{CaballeroHill1992}
Caballero, A. and W.~G. Hill. 1992.
\newblock Effects of partial inbreeding on fixation rates and variation of mutant genes.
\newblock Genetics 131:493--507.

\bibitem[{Fry(2010)Fry}]{Fry2010}
Fry, J.~D. 2010.
\newblock The genomic location of sexually antagonistic variation: some cautionary comments.
\newblock Evolution 64:1510-1516.

\bibitem[{Holden(1979)Holden}]{Holden1979}
Holden, L.~R. 1979.
\newblock New properties of the two-locus partial selfing model with selection.
\newblock Genetics 93:217--236.

\bibitem[{Jordan and Connallon(2014)Jordan and Connallon}]{JordanConnallon2014}
Jordan, C.~Y. and T. Connallon. 2014.
\newblock Sexually antagonistic polymorphism in simultaneous hermaphrod-\\ites.
\newblock Evolution 68:3555--3569.

\bibitem[{Kidwell et~al.(1977)Kidwell, Clegg, Stewart, and Prout}]{Kidwell1977}
Kidwell, J.~F., M.~T. Clegg, F.~M. Stewart, and T. Prout. 1977.
\newblock Regions of stable equilibria for models of differential selection in the two sexes under random mating.
\newblock Genetics 85:171--183.

\bibitem[{Nagylaki(1997)Nagylaki}]{Nagylaki1997}
Nagylaki, T. 1997.
\newblock The diffusion model for migration and selection in a plant population.
\newblock Journal of Mathematical Biology 35:409--431.

\bibitem[{Otto and Day(2007)Otto and Day}]{OttoDay2007}
Otto, S.~P. and T. Day. 2007.
\newblock A biologist's guide to mathematical modeling in ecology and evolution.
\newblock Princeton University Press, Princeton, New Jersey, USA.

\bibitem[{Patten et~al.(2010)Patten et~al.}]{Patten2010}
M.~M. Patten, D. Haig, and F. \'{U}beda. 2010.
\newblock Fitness variation due to sexual antagonism and linkage disequilibrium.
\newblock Evolution 64:3638--3642.

\bibitem[{Prout(1968)Prout}]{Prout1968}
Prout, T. 1968.
\newblock Sufficient conditions for multiple niche polymorphism.
\newblock American Naturalist 102:493--496.

\bibitem[{Prout(2000)Prout}]{Prout2000}
Prout, T. 2000.
\newblock How well does opposing selection maintain variation?
\newblock Pp. 157--203 \textit{in} R.~S. Singh and C.~B. Kimbras, eds. Evolutionary Genetics from molecules to morphology. Cambridge University Press, Cambridge, U.K.




\end{thebibliography}

\end{document}
